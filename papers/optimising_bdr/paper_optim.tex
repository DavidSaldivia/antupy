% Template for manual conversion from DOCX to LaTeX
% Project: Beam-Down Receiver (BDR) optimisation paper
% Notes:
% - Use this as a clean starting point; copy/paste text from Word and then
%   convert formatting to LaTeX commands.
% - Keep labels consistent and use \ref, \eqref, and \cite throughout.

\documentclass[11pt,a4paper]{article}

% Encoding, fonts, language
\usepackage[utf8]{inputenc}
\usepackage[T1]{fontenc}
\usepackage[english]{babel}
\usepackage{csquotes}

% Math & symbols
\usepackage{amsmath, amssymb}
\usepackage{siunitx}
\sisetup{detect-all}

% Graphics & floats
\usepackage{graphicx}
\usepackage{caption}
\usepackage{subcaption}
\usepackage{booktabs}
\usepackage{longtable}
\usepackage{multirow}
\usepackage{array}
\usepackage{float}

% Links
\usepackage[hidelinks]{hyperref}

% Misc utilities
\usepackage{xcolor}
\usepackage{soul} % \hl{} temporary highlighting during conversion

% Metadata
\title{Optimizing a Modular Beam-Down Concentrated Solar Power Concept}
\author{David Saldivia \and Jose Bilbao \and Anna Bruce \and Robert A. Taylor}
\date{Draft: \today}

% Helper macros
\newcommand{\bdtp}{BD-TPR}

\begin{document}

\maketitle

\begin{abstract}
% Paste abstract from Word and tidy up markup. Keep plain text first; then add math/refs.
This template provides a clean starting point to convert a DOCX document into LaTeX for the \bdtp{} optimisation study. Replace this text with your abstract.
\end{abstract}

\tableofcontents
\clearpage

% -------------------- INTRO --------------------
\section{Introduction}\label{sec:intro}
% Paste text; replace manual formatting with LaTeX commands.
% Use \textbf{}, \textit{}, and LaTeX lists instead of Word formatting.
Motivation and context. Transition to high shares of renewable energy and the role of thermal storage. Prior BDR and particle receiver work. Contributions of this paper.

% Example citation: \cite{saldivia2021optical}

% -------------------- METHODS --------------------
\section{Methods}\label{sec:methods}

\subsection{Optimisation Framework}\label{sec:opt-framework}
Brief overview of the optimisation loop and modules (optics, receiver, costing).

\begin{figure}[H]
  \centering
  % Replace with your image path
  % \includegraphics[width=0.9\linewidth]{media/flow.png}
  \fbox{\rule{0pt}{2in}\rule{0.9\linewidth}{0pt}}% placeholder box
  \caption{Optimisation algorithm flowchart.}
  \label{fig:flow}
\end{figure}

\subsection{Cost Model}\label{sec:cost-model}
We compute the Levelized Cost of Heat ($LCOH$) using:
\begin{equation}
  \label{eq:lcoh}
  LCOH = \frac{C_{C}\,\bigl(1 + C_{O\&M}\,TPF\bigr)}{TPF\, P_{\mathrm{yr}}}
\end{equation}
with the total payment factor (TPF):
\begin{equation}
  \label{eq:tpf}
  TPF = \frac{1}{DR}\left(1 - \frac{1}{(1+DR)^N}\right).
\end{equation}
Define all symbols after the equations.

\subsection{Optical and Thermal Models}\label{sec:models}
High-level description; refer to previous validated modules.

% -------------------- RESULTS --------------------
\section{Results}\label{sec:results}

\subsection{LCOH Minimisation}\label{sec:lcoh-min}
Key plots and summary of trends.

\begin{figure}[H]
  \centering
  % \includegraphics[width=0.49\linewidth]{media/plot_a.png}\hfill
  % \includegraphics[width=0.49\linewidth]{media/plot_b.png}
  \fbox{\rule{0pt}{1.6in}\rule{0.49\linewidth}{0pt}}\hfill
  \fbox{\rule{0pt}{1.6in}\rule{0.49\linewidth}{0pt}}
  \caption{LCOH minimisation across receiver power, tower height, and flux levels.}
  \label{fig:lcoh-min}
\end{figure}

% Example labeled equations derived from fits
\begin{align}
  \label{eq:prcv}
  P_{\mathrm{rcv}} &= 4.0\, e^{0.029\, z_{f}},\\
  \label{eq:qavg}
  Q_{\mathrm{avg}} &= 1.54 - 1.93\, e^{-0.047\, z_{f}},\\
  \label{eq:fzv}
  f_{zv} &= 0.78 + 0.24\, e^{-0.040\, z_{f}},\\
  \label{eq:lcoh-vs-zf}
  LCOH &= 22.0 + 32.6\, e^{-0.079\, z_{f}}.
\end{align}

\subsection{Technical Parameters}\label{sec:technical}
Describe final selected case and include panel of figures if available.

\begin{table}[H]
  \centering
  \caption{Summary of key technical parameters for the selected design.}
  \label{tab:tech-summary}
  \begin{tabular}{@{}ll@{}}
    \toprule
    Parameter & Value \\
    \midrule
    Tower height, $z_f$ & \SI{50}{m} \\
    Receiver power, $P_{rcv}$ & \SI{19}{MW_{th}} \\
    Mean radiation flux, $Q_{avg}$ & \SI{1.25}{MW.m^{-2}} \\
    Vertex ratio, $f_{zv}$ & 0.818 \\
    Receiver efficiency & \SI{83.1}{\percent} \\
    Solar field efficiency & \SI{63.4}{\percent} \\
    \bottomrule
  \end{tabular}
\end{table}

\subsection{Cost Distribution and Sensitivity}\label{sec:cost-sens}
Pie/bar charts for costs; tornado plots for sensitivities.

\subsection{Annual Performance}\label{sec:annual}
Monthly energy flows and efficiencies; capacity factor.

% -------------------- CONCLUSION --------------------
\section{Conclusion}\label{sec:conclusion}
Main findings and recommendations. Future work.

% -------------------- ACKS & DATA --------------------
\section*{Acknowledgements}
% Optional

\section*{Data and Code Availability}
% Optional

% -------------------- REFERENCES --------------------
\bibliographystyle{ieeetr}
\bibliography{../dwh/solarshift}

\end{document}
