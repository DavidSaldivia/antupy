\documentclass[11pt]{article}
\usepackage[utf8]{inputenc}
\usepackage[T1]{fontenc}
\usepackage{amsmath,amssymb}
\usepackage{graphicx}
\usepackage{booktabs}
\usepackage{longtable}
\usepackage{array}
\usepackage{multirow}
\usepackage{soul} % for \hl{}
\usepackage{hyperref}

\title{Optimizing a Modular Beam-Down Concentrated Solar Power Concept}
\author{David Saldivia \and Jose Bilbao \and Anna Bruce \and Robert A. Taylor}
\date{\today}

\begin{document}

\maketitle
\tableofcontents
\clearpage

\section{Introduction}\label{introduction}

What should go here?

\begin{itemize}
\item
  Something about transition to 100\% renewable and the need for
  storage.
\item
  CSP and modular designs.
\item
  What have been done with BDR, solid particles.
\item
  Something about markets, storage value and the kind.
\end{itemize}

\section{Methods}\label{methods}

\subsection{Introduction}\label{introduction-1}

\subsection{Optimization Tool}\label{optimization-tool}

\subsection{Optical and Thermal
models}\label{optical-and-thermal-models}

The present paper is a final report xxxxxxxxx

\begin{figure}
\centering
\includegraphics[width=6.14488in,height=2.65893in]{media/image1.png}
\caption{Optimisation algorithm flow-chart. In blue the
components developed specifically for the opimisation. In yellow the
functions from the optical simulation and in green the functions from
the receiver simulation.}
\label{fig:optimisation-flow}
\end{figure}

\hl{Write three paragraphs:}

\begin{equation}\label{eq:prcv}
P_{rcv} = 4.0\, e^{0.029\, z_{f}}
\end{equation}

\begin{equation}\label{eq:qavg}
Q_{avg} = 1.54 - 1.93\, e^{- 0.047\, z_{f}}
\end{equation}

\begin{equation}\label{eq:fzv}
f_{zv} = 0.78 + 0.24\, e^{- 0.040\, z_{f}}
\end{equation}

\begin{equation}\label{eq:lcoh_vs_zf}
LCOH = 22.0 + 32.6\, e^{- 0.079\, z_{f}}
\end{equation}
\def\labelenumi{\arabic{enumi}.}
\item
  \hl{One about the different modules that were written. Refer the
  reader to the Appendix.}
\item
  \hl{One about the independent variables. Refer the reader to the
  Appendix.}
\item
  \hl{One about the detailed discussion of the optimization. Refer the
  redear to the Appendix.}
\end{enumerate}

\subsection{Power Block and Dispatch
Model}\label{power-block-and-dispatch-model}

\subsection{Costings}\label{costings}

The costing module was built using project assessment metrics and a
combination of conventional values frequently used in CSP financial
assessments in combination with specific cost estimations for BDRs and
particle-based systems. The Levelized Cost of Heat was calculated using
the well-known formula for future constant payments:

\begin{equation}
\label{eq:lcoh}
LCOH = \frac{C_{C}\left( 1 + C_{O\& M}\, TPF \right)}{TPF\, P_{yr}}
\end{equation}

\begin{equation}
\label{eq:tpf}
TPF = \frac{1}{DR}\left( 1 - \frac{1}{(1 + DR)^{N}} \right)
\end{equation}

Where \(C_{C}\) is the capital cost, including land, heliostat, BDR
mirrors, tower, receiver, storage, engineering, and contingency costs.
\(C_{O\& M}\) is the annual operation and maintenance costs. \(P_{yr}\)
is the annual heat generation and \(TPF\) is the total payment factor
for a given discount rate (\(DR\)), and plant lifespan (\(N\)).

The capital costs are divided into two categories: conventional and
specific parameters. The conventional parameters are those that are
independent of the type of CSP plant, and therefore, common values in
the literature are used. The specific parameters are those specific to
solid particle-based systems with beam-down optics. For some of these,
literature values were available, while for others some estimations were
made.

\hl{Write one or two paragraphs to refer the reader to the Supplementary
Documents}

\hl{Present a Summary Table}

\subsection{Market Data}\label{market-data}

\section{Results}\label{results}

The results are divided into two sections. First, the main results from
the optimisation are presented. For this, the LCOH minimisation follows
an iterative process, obtaining a range of design parameter at the end.
Then, for a selected case, additional detailed results are presented,
which exemplify the extension of detailed analysis that is possible to
carry on with the modules developed through this Thesis. This includes
technical and economic results. Finally, a sensitivity analysis and
annual performance for the selected case are presented.

\subsection{Minimisation of LCOH}\label{minimisation-of-lcoh}

The results for the solar thermal subsystem optimisation are shown in
Figure 6.10, with Figure 6.10 a) presenting the LCOH as a function of
the receiver power for different tower heights (20 m to 75 m). The
minimum value for each tower height is presented as a blue line. The
same min LCOH line is shown in Figure 6.10 b) plus the corresponding
receiver power (red squares). In these two subplots, an average
radiation flux of 1.0 MW\textsubscript{th}/m\textsuperscript{2} was
used. The results for 0.5 MW\textsubscript{th}/m\textsuperscript{2} and
1.5 MW\textsubscript{th}/m\textsuperscript{2} are presented in subplots
c) and d), respectively. In both cases, it was observed that a minimum
LCOH lower than 24 USD/MW\textsubscript{th} was obtained for a tower
height above 35 m. This indicates that initially, there are benefits to
increasing tower height; however, the benefits diminish until a
plateaued zone (above 65 m), where there are practically no benefits to
increase tower height. The main reasons for this are the increased tower
costs and lower BDR optical efficiency for taller towers.

On the other hand, the average radiation flux has a limited influence
over the LCOH. For example, for a 50 m tower, increasing the radiation
flux from 0.5 to 1.5 MW\textsubscript{th}/m\textsuperscript{2} reduces
the LCOH from 26~USD/MWh to 23 MW/m\textsuperscript{2}. However, the
average radiation flux affects the optimal receiver power. For the same
height, the optimal power increases from \textasciitilde12
MW\textsubscript{th} to \textasciitilde18 MW\textsubscript{th} if the
radiation flux increases from 0.5 to 1.5
MW\textsubscript{th}/m\textsuperscript{2}. Therefore, the influence of
average radiation flux is analysed in more detail.

\begin{longtable}[]{@{}
  >{\raggedright\arraybackslash}p{(\linewidth - 2\tabcolsep) * \real{0.5031}}
  >{\raggedright\arraybackslash}p{(\linewidth - 2\tabcolsep) * \real{0.4969}}@{}}
\caption{Result of LCOH minimisation for different receiver
power, tower height, and radiation fluxes.}\tabularnewline
\toprule\noalign{}
\begin{minipage}[b]{\linewidth}\raggedright
\includegraphics[width=3.21334in,height=1.81278in]{media/image2.png}
\end{minipage} & \begin{minipage}[b]{\linewidth}\raggedright
\includegraphics[width=3.14219in,height=1.87864in]{media/image3.png}
\end{minipage} \\
\midrule\noalign{}
\endfirsthead
\toprule\noalign{}
\begin{minipage}[b]{\linewidth}\raggedright
\includegraphics[width=3.21334in,height=1.81278in]{media/image2.png}
\end{minipage} & \begin{minipage}[b]{\linewidth}\raggedright
\includegraphics[width=3.14219in,height=1.87864in]{media/image3.png}
\end{minipage} \\
\midrule\noalign{}
\endhead
\bottomrule\noalign{}
\endlastfoot
\begin{minipage}[t]{\linewidth}\raggedright
\begin{enumerate}
\def\labelenumi{\alph{enumi})}
\item
  Results for different receiver powers and tower heights
  (Q\textsubscript{avg}=1MW/m\textsuperscript{2})
\end{enumerate}
\end{minipage} & \begin{minipage}[t]{\linewidth}\raggedright
\begin{enumerate}
\def\labelenumi{\alph{enumi})}
\setcounter{enumi}{1}
\item
  LCOH curve with selected conditions from a).
\end{enumerate}
\end{minipage} \\
\includegraphics[width=3.10531in,height=1.85659in]{media/image4.png} &
\includegraphics[width=3.17175in,height=1.86744in]{media/image5.png} \\
\begin{minipage}[t]{\linewidth}\raggedright
\begin{enumerate}
\def\labelenumi{\alph{enumi})}
\setcounter{enumi}{2}
\item
  LCOH curve for Q\textsubscript{avg}=0.5MW/m\textsuperscript{2}
\end{enumerate}
\end{minipage} & \begin{minipage}[t]{\linewidth}\raggedright
\begin{enumerate}
\def\labelenumi{\alph{enumi})}
\setcounter{enumi}{3}
\item
  LCOH curve for Q\textsubscript{avg}=1.5MW/m\textsuperscript{2}
\end{enumerate}
\end{minipage} \\
\end{longtable}

For each tower height, the same procedure is repeated for different
average radiation flux values ranging from 0.5 to 2.0
MW\textsubscript{th}/m\textsuperscript{2}. The result from this process
is presented in Figure 6.11 for four different heights: a) 30 m, b) 40
m, c) 50 m, and d) 60 m. In each case, a minimum exists for a
combination of radiation flux and receiver power, which is indicated
with a red star in each subplot. Both, the optimal receiver power, and
the radiation flux increase with tower height. For a 30 m tower it is
around 9 MWth and 1.0 MW\textsubscript{th}/m\textsuperscript{2}, while
for a 60 m tower it is around 26 MW\textsubscript{th} and 1.5
MW\textsubscript{th}/m\textsuperscript{2}. The radiation flux is optimal
for all the tower heights in the range of 1.0 to 1.5
MW\textsubscript{th}/m\textsuperscript{2}. For short towers there is a
single optimal point, but for towers above 50 m, the bottom part of the
`U-shaped' curve is relatively flat, so there is a wide range of
possible combinations. For example, in a 50 m tower, receiver power
between 12 and 25 MW\textsubscript{th} with radiation fluxes between
1.25 and 1.5 MW\textsubscript{th}/m\textsuperscript{2} have a similar
LCOH of \textasciitilde23 USD/MW\textsubscript{th.}

\begin{longtable}[]{@{}
  >{\raggedright\arraybackslash}p{(\linewidth - 4\tabcolsep) * \real{0.4163}}
  >{\raggedright\arraybackslash}p{(\linewidth - 4\tabcolsep) * \real{0.4111}}
  >{\raggedright\arraybackslash}p{(\linewidth - 4\tabcolsep) * \real{0.1726}}@{}}
\caption{LCOH as a function of receiver power for different
radiation fluxes and tower heights. In each case the minimum optimal
value is presented with a red star.}\tabularnewline
\toprule\noalign{}
\begin{minipage}[b]{\linewidth}\centering
\includegraphics[width=2.84493in,height=1.94404in]{media/image6.png}

\begin{enumerate}
\def\labelenumi{\alph{enumi})}
\item
  \(z_{f} = 30m\)
\end{enumerate}
\end{minipage} & \begin{minipage}[b]{\linewidth}\centering
\includegraphics[width=2.83696in,height=1.94484in]{media/image7.png}

\begin{enumerate}
\def\labelenumi{\alph{enumi})}
\setcounter{enumi}{1}
\item
  \(z_{f} = 40m\)
\end{enumerate}
\end{minipage} &
\multirow{2}{=}{\begin{minipage}[b]{\linewidth}\centering
\includegraphics[width=1.10717in,height=1.58944in]{media/image8.png}
\end{minipage}} \\
\begin{minipage}[b]{\linewidth}\centering
\includegraphics[width=2.86543in,height=1.93426in]{media/image9.png}

\begin{enumerate}
\def\labelenumi{\alph{enumi})}
\setcounter{enumi}{2}
\item
  \(z_{f} = 50m\)
\end{enumerate}
\end{minipage} & \begin{minipage}[b]{\linewidth}\centering
\includegraphics[width=2.76477in,height=1.88753in]{media/image10.png}

\begin{enumerate}
\def\labelenumi{\alph{enumi})}
\setcounter{enumi}{3}
\item
  \(z_{f} = 60m\)
\end{enumerate}
\end{minipage} \\
\midrule\noalign{}
\endfirsthead
\toprule\noalign{}
\begin{minipage}[b]{\linewidth}\centering
\includegraphics[width=2.84493in,height=1.94404in]{media/image6.png}

\begin{enumerate}
\def\labelenumi{\alph{enumi})}
\item
  \(z_{f} = 30m\)
\end{enumerate}
\end{minipage} & \begin{minipage}[b]{\linewidth}\centering
\includegraphics[width=2.83696in,height=1.94484in]{media/image7.png}

\begin{enumerate}
\def\labelenumi{\alph{enumi})}
\setcounter{enumi}{1}
\item
  \(z_{f} = 40m\)
\end{enumerate}
\end{minipage} &
\multirow{2}{=}{\begin{minipage}[b]{\linewidth}\centering
\includegraphics[width=1.10717in,height=1.58944in]{media/image8.png}
\end{minipage}} \\
\begin{minipage}[b]{\linewidth}\centering
\includegraphics[width=2.86543in,height=1.93426in]{media/image9.png}

\begin{enumerate}
\def\labelenumi{\alph{enumi})}
\setcounter{enumi}{2}
\item
  \(z_{f} = 50m\)
\end{enumerate}
\end{minipage} & \begin{minipage}[b]{\linewidth}\centering
\includegraphics[width=2.76477in,height=1.88753in]{media/image10.png}

\begin{enumerate}
\def\labelenumi{\alph{enumi})}
\setcounter{enumi}{3}
\item
  \(z_{f} = 60m\)
\end{enumerate}
\end{minipage} \\
\midrule\noalign{}
\endhead
\bottomrule\noalign{}
\endlastfoot
\end{longtable}

If the minimum LCOH points for different towers are presented together,
then a design curve for Beam-Down Tilted-Particle Receivers (BD-TPR) is
generated. Figure 6.12 a) presents these results for towers between 20 m
and 75 m. This final curve presents the results once the four
independent variables are optimised. For a given tower height, the
optimal receiver power is shown in red squares and the minimum expected
LCOH (in USD/MW\textsubscript{th}) is presented in blue circles. The
optimal \(Q_{avg}\) and \(f_{zv}\) as function of tower height are also
presented in Figure 6.12 b) and c), respectively.

\begin{longtable}[]{@{}
  >{\raggedright\arraybackslash}p{(\linewidth - 2\tabcolsep) * \real{0.6634}}
  >{\raggedright\arraybackslash}p{(\linewidth - 2\tabcolsep) * \real{0.3366}}@{}}
\caption{Final function for minimum LCOH as function of
tower height. a) minimum LCOH and optimal receiver power, b) optimal
average radiation flux, c) optimal vertex ratio.}\tabularnewline
\toprule\noalign{}
\multirow{2}{=}{\begin{minipage}[b]{\linewidth}\centering
\includegraphics[width=4.10471in,height=2.71492in]{media/image11.png}

\begin{enumerate}
\def\labelenumi{\alph{enumi})}
\item
  \(LCOH\) and \(P_{rcv}\)
\end{enumerate}
\end{minipage}} & \begin{minipage}[b]{\linewidth}\centering
\includegraphics[width=1.9929in,height=1.33019in]{media/image12.png}

\begin{quote}
b) \(Q_{avg}\)
\end{quote}
\end{minipage} \\
& \begin{minipage}[b]{\linewidth}\centering
\includegraphics[width=2.00732in,height=1.32085in]{media/image13.png}

c) \(f_{zv}\)
\end{minipage} \\
\midrule\noalign{}
\endfirsthead
\toprule\noalign{}
\multirow{2}{=}{\begin{minipage}[b]{\linewidth}\centering
\includegraphics[width=4.10471in,height=2.71492in]{media/image11.png}

\begin{enumerate}
\def\labelenumi{\alph{enumi})}
\item
  \(LCOH\) and \(P_{rcv}\)
\end{enumerate}
\end{minipage}} & \begin{minipage}[b]{\linewidth}\centering
\includegraphics[width=1.9929in,height=1.33019in]{media/image12.png}

\begin{quote}
b) \(Q_{avg}\)
\end{quote}
\end{minipage} \\
& \begin{minipage}[b]{\linewidth}\centering
\includegraphics[width=2.00732in,height=1.32085in]{media/image13.png}

c) \(f_{zv}\)
\end{minipage} \\
\midrule\noalign{}
\endhead
\bottomrule\noalign{}
\endlastfoot
\end{longtable}

Curve fittings were performed over these relationships, which
constitutes correlations to design a BD-TPR plant. For a given tower
height, the optimal receiver power (Equation(\emph{6}.\emph{26})),
radiation flux (Equation (\emph{6}.\emph{27})), vertex ratio (Equation
(\emph{6}.\emph{28})) and an expected LCOH (Equation
(\emph{6}.\emph{29})) can be calculated.
\[P_{rcv} = 4.0e^{0.029\ z_{f}}\]
\end{minipage} & \begin{minipage}[b]{\linewidth}\centering
(6.26)
\end{minipage} \\
\midrule\noalign{}
\endhead
\bottomrule\noalign{}
\endlastfoot
\(Q_{avg} = 1.54 - 1.93e^{- 0.047\ z_{f}}\) & (6.27) \\
\(f_{zv} = 0.78 + 0.24e^{- 0.040\ z_{f}}\) & (6.28) \\
\(LCOH = 22.0 + 32.6e^{- 0.079\ z_{f}}\) & (6.29) \\
\end{longtable}
}

Considering that this design is intended for power generation, where
larger power blocks are significantly more cost-effective, a minimum
receiver power of 10 MW\textsubscript{th} was considered. On other hand,
although the results presented are considering towers up to 75 m, taller
towers pose additional issues. For example, large hyperboloid mirrors
(e.g., 20 m radius) could start facing wind and structural issues. For
taller towers, the alignment errors become more important and,
therefore, the optical efficiencies probably will be lower than the one
presented here. For these reasons, the recommended design range for
BD-TPR presented here is for towers between 35 m to 50 m, with receiver
capacities of 10 MW\textsubscript{th} and 20 MW\textsubscript{th},
respectively. In summary, a tower height of
\(\mathbf{z}_{\mathbf{f}}\mathbf{= 50\ m}\), with a receiver power of
\(\mathbf{P}_{\mathbf{rcv}}\mathbf{= 19\ M}\mathbf{W}_{\mathbf{th}}\)
and a radiation flux of
\(\mathbf{Q}_{\mathbf{avg}}\mathbf{= 1.25}\frac{\mathbf{M}\mathbf{W}_{\mathbf{th}}}{\mathbf{m}^{\mathbf{2}}}\)
was found to obtain a
\(\mathbf{\min}\left( \mathbf{LCOH} \right)\mathbf{= 23.01\ USD/MWh}\).
Now that the overall efficiency is presented and the main relationships
between the independent variables and the LCOH were stated, a detailed
analysis for the selected design parameters is presented in the next
subsection.

\subsubsection{Technical parameters}\label{technical-parameters}

Detailed results for the final selected case are presented in this
Subsection to illustrate the capabilities of the module developed so
far. As this is the final selected case and one of the main
recommendations from this thesis, the main components of the whole plant
are presented, covering results from most of the previous Chapters.
Figure 6.13 shows the final results for the most important components in
a BD-TPR design. Figure 6.13 a) presents the final heliostat field of
4570 heliostats with a design point optical efficiency of 64.1\%. Due to
the latitude (-25°) the solar field has a slightly polar configuration,
which highlights one of the advantages of BDR optics. That is, it can
utilise a compact surround tower with a solar field extended angle of
360°, even for a cavity receiver. The heliostats range up to 150 m north
of the receiver location and 300 m south of it. The solar field is
symmetrical in its E-W axis, with 300 m extent in each direction. Figure
6.13 b) shows the hyperboloid radiation flux. A maximum value of around
60 kW/m\textsuperscript{2} is found. The inner and outer radii are 5.1 m
and 20.3 m, with a total 1316 m\textsuperscript{2}. The vertex ratio is
0.8181, and the hyperboloid mirror ranges between 43 m and 50 m heights.

\begin{longtable}[]{@{}
  >{\centering\arraybackslash}p{(\linewidth - 2\tabcolsep) * \real{0.5235}}
  >{\raggedright\arraybackslash}p{(\linewidth - 2\tabcolsep) * \real{0.4765}}@{}}
\caption{Detailed results for the selected design. a)
heliostat field, b) radiation flux over hyperboloid mirror, c) radiation
flux profile on receiver aperture, d) radiation flux on the tilted
surface of the receiver (only one of three receivers showed here), e)
specific efficiency on receiver surface, f) particle temperature
distribution on receiver surface, g) particle temperature distribution
on receiver outlet, h) outlet temperature by percentile.}\tabularnewline
\toprule\noalign{}
\begin{minipage}[b]{\linewidth}\centering
\includegraphics[width=3.12791in,height=2.1108in]{media/image14.png}

\begin{enumerate}
\def\labelenumi{\alph{enumi})}
\item
  Heliostat field
\end{enumerate}
\end{minipage} & \begin{minipage}[b]{\linewidth}\centering
\includegraphics[width=2.44186in,height=1.8832in]{media/image15.png}

\begin{enumerate}
\def\labelenumi{\alph{enumi})}
\setcounter{enumi}{1}
\item
  Radiation flux in hyperboloid mirror
\end{enumerate}
\end{minipage} \\
\midrule\noalign{}
\endfirsthead
\toprule\noalign{}
\begin{minipage}[b]{\linewidth}\centering
\includegraphics[width=3.12791in,height=2.1108in]{media/image14.png}

\begin{enumerate}
\def\labelenumi{\alph{enumi})}
\item
  Heliostat field
\end{enumerate}
\end{minipage} & \begin{minipage}[b]{\linewidth}\centering
\includegraphics[width=2.44186in,height=1.8832in]{media/image15.png}

\begin{enumerate}
\def\labelenumi{\alph{enumi})}
\setcounter{enumi}{1}
\item
  Radiation flux in hyperboloid mirror
\end{enumerate}
\end{minipage} \\
\midrule\noalign{}
\endhead
\bottomrule\noalign{}
\endlastfoot
\begin{minipage}[t]{\linewidth}\centering
\includegraphics[width=2.69758in,height=1.99955in]{media/image16.png}

\begin{enumerate}
\def\labelenumi{\alph{enumi})}
\setcounter{enumi}{2}
\item
  Radiation flux on receiver aperture
\end{enumerate}
\end{minipage} & \begin{minipage}[t]{\linewidth}\raggedright
\includegraphics[width=2.38372in,height=2.1978in]{media/image17.png}

\begin{enumerate}
\def\labelenumi{\alph{enumi})}
\setcounter{enumi}{3}
\item
  Radiation flux on tilted surface
\end{enumerate}
\end{minipage} \\
\begin{minipage}[t]{\linewidth}\centering
\includegraphics[width=2.18605in,height=1.98837in]{media/image18.png}

\begin{enumerate}
\def\labelenumi{\alph{enumi})}
\setcounter{enumi}{4}
\item
  Specific efficiency
\end{enumerate}
\end{minipage} & \begin{minipage}[t]{\linewidth}\centering
\includegraphics[width=2.11682in,height=1.89583in]{media/image19.png}

\begin{enumerate}
\def\labelenumi{\alph{enumi})}
\setcounter{enumi}{5}
\item
  Particles temperature
\end{enumerate}
\end{minipage} \\
\begin{minipage}[t]{\linewidth}\centering
\includegraphics[width=2.35679in,height=1.54438in]{media/image20.png}

\begin{enumerate}
\def\labelenumi{\alph{enumi})}
\setcounter{enumi}{6}
\item
  Outlet particles temperature profile
\end{enumerate}
\end{minipage} & \begin{minipage}[t]{\linewidth}\centering
\includegraphics[width=2.28342in,height=1.4963in]{media/image21.png}

\begin{enumerate}
\def\labelenumi{\alph{enumi})}
\setcounter{enumi}{7}
\item
  Outlet temperature distribution
\end{enumerate}
\end{minipage} \\
\end{longtable}

Figure 6.13 c) shows the radiation flux in the receiver aperture for the
selected TOD mirror consisting of 3-hexagonal paraboloid mirror with a
concentration ratio of 2.0. The TOD height is 1.73 m, with a mirror
surface area of 60.79 m\textsuperscript{2}. The maximum radiation flux
is up to 3 MW\textsubscript{th}/m\textsuperscript{2}; however, it hits
the virtual aperture surface defined by the outlet of the TOD. The
radiation flux is spread inside the receiver's cavity and is
redistributed. Figure 6.13 d) shows the radiation flux in the final
tilted particle receiver (only one of the three receivers is shown, and
the tilted plane is presented). The maximum radiation flux is 2.55
MW\textsubscript{th}/m\textsuperscript{2}, which is still within the
maximum radiation flux acceptable for ceramics
(\textasciitilde3MW\textsubscript{th}/m\textsuperscript{2}). Figure 6.13
e) and f) show the specific efficiency and particle temperature map on
the tilted surface. An overall 83.3\% efficiency is found, which is in
the range of expected values for particle receivers (50-90\%) {[}20{]}.
Finally, Figure 6.13 g) and h) show the final distribution at the outlet
of the receiver. Subplot g) shows the temperature distribution at a
cross-section in the receiver\textquotesingle s outlet. In contrast,
subplot h) shows the same information in increasing order, showing a
relatively linear distribution of temperature among the particles. The
particle distribution ranges between 1039 K to 1285 K, which is
explained by the uneven radiation flux on the receiver (as shown in
Figure 6.13 d). This undesired result could be improved with a heliostat
control strategy to obtain more homogeneous radiation flux, as was
discussed in Chapter 5. Finally, Table 6.5 presents additional detailed
information for the selected case.

\begin{longtable}[]{@{}
  >{\raggedright\arraybackslash}p{(\linewidth - 6\tabcolsep) * \real{0.2989}}
  >{\raggedright\arraybackslash}p{(\linewidth - 6\tabcolsep) * \real{0.1733}}
  >{\raggedright\arraybackslash}p{(\linewidth - 6\tabcolsep) * \real{0.2836}}
  >{\raggedright\arraybackslash}p{(\linewidth - 6\tabcolsep) * \real{0.2364}}@{}}
\caption{Table 6.5. Detailed results for the optimised BD-TPR design.
Bold values are the independent variables that were
optimised.}\tabularnewline
\toprule\noalign{}
\begin{minipage}[b]{\linewidth}\raggedright
Parameter
\end{minipage} & \begin{minipage}[b]{\linewidth}\centering
Value
\end{minipage} & \begin{minipage}[b]{\linewidth}\raggedright
Parameter
\end{minipage} & \begin{minipage}[b]{\linewidth}\centering
Value
\end{minipage} \\
\midrule\noalign{}
\endfirsthead
\toprule\noalign{}
\begin{minipage}[b]{\linewidth}\raggedright
Parameter
\end{minipage} & \begin{minipage}[b]{\linewidth}\centering
Value
\end{minipage} & \begin{minipage}[b]{\linewidth}\raggedright
Parameter
\end{minipage} & \begin{minipage}[b]{\linewidth}\centering
Value
\end{minipage} \\
\midrule\noalign{}
\endhead
\bottomrule\noalign{}
\endlastfoot
\textbf{Tower height,} \(\mathbf{z}_{\mathbf{f}}\) & \(\mathbf{50\ m}\)
& Receiver efficiency & \(83.1\%\) \\
\textbf{Receiver thermal power,} \(\mathbf{P}_{\mathbf{rcv}}\) &
\(\mathbf{19\ M}\mathbf{W}_{\mathbf{th}}\) & Solar field efficiency &
\(63.4\%\) \\
\textbf{Receiver mean radiation flux,} \(\mathbf{Q}_{\mathbf{avg}}\) &
\(\mathbf{1.25\ MW/}\mathbf{m}^{\mathbf{2}}\) & Geometric concentration
ratio & \(2105( - )\) \\
\textbf{Vertex ratio,} \(\mathbf{f}_{\mathbf{zv}}\) &
\(\mathbf{0.818( - )}\) & Receiver max radiation flux &
\(2.56\ MW/m^{2}\) \\
Receiver area & \(18.06\ m^{2}\) & Particle temp. range &
\(1039 - 1285\ K\) \\
Number of heliostats & \(4456( - )\) & Mass flow rate &
\(59.32\ kg/s\) \\
HB mirror radius & \(20.13\ m\) & Mean residence time & \(27.69\ s\) \\
HB mirror surface & \(1316.5\ m^{2}\) & Storage volume &
\(471.8m^{3}\) \\
SF mirror surface & \(38010.6\ m^{2}\) & Capacity Factor &
\(23.13\%\) \\
TOD mirror surface & \(60.79\ m^{2}\) & Land productivity &
\(0.98\ MW_{th}/ha\) \\
TOD height & \(1.73\ m\) & LCOH & \(23.01\ USD/MW_{th}\) \\
Total mirror surface & \(39387.9\ m^{2}\) & LCOH (AUD) &
\(32.21\ AUD/MW_{th}\) \\
\end{longtable}

\subsubsection{Cost distribution}\label{cost-distribution}

For the selected case, the capital cost distribution is presented in
Figure 6.14 a). To put this number in context, a conventional case is
also presented using the default conditions given by SolarPILOT for a
large plant (195 m tower and 670 MW\textsubscript{th} for solar field).
The comparison is done with a conventional plant with large capacity,
because the default costs for tower and receiver would have generated
unrealistic high values for short towers/small receivers. The only
component that SolarPILOT does not calculate is the storage cost, so the
same BD plant cost estimation was used (see Section 6.3.3). Finally, the
heliostat cost is set at 100 USD/m\textsuperscript{2}, instead of the
default 140 USD/m\textsuperscript{2}, to reflect the expected cost
reductions discussed in Section 6.3.2. Financial parameters (discount
rate, lifetime, EPC costs, etc) are kept the same in both cases.

\begin{longtable}[]{@{}
  >{\raggedright\arraybackslash}p{(\linewidth - 2\tabcolsep) * \real{0.5174}}
  >{\raggedright\arraybackslash}p{(\linewidth - 2\tabcolsep) * \real{0.4826}}@{}}
\caption{Capital Costs Distributions. a) BD-TPR plant, b)
conventional plant.}\tabularnewline
\toprule\noalign{}
\begin{minipage}[b]{\linewidth}\raggedright
\includegraphics[width=2.97638in,height=2.40157in]{media/image22.png}

\begin{enumerate}
\def\labelenumi{\alph{enumi})}
\item
  BD-TPR plant (LCOH=23.01 USD/MWh)
\end{enumerate}
\end{minipage} & \begin{minipage}[b]{\linewidth}\raggedright
\includegraphics[width=2.87008in,height=2.40157in]{media/image23.png}

b) conventional (LCOH=26.53 USD/MWh)
\end{minipage} \\
\midrule\noalign{}
\endfirsthead
\toprule\noalign{}
\begin{minipage}[b]{\linewidth}\raggedright
\includegraphics[width=2.97638in,height=2.40157in]{media/image22.png}

\begin{enumerate}
\def\labelenumi{\alph{enumi})}
\item
  BD-TPR plant (LCOH=23.01 USD/MWh)
\end{enumerate}
\end{minipage} & \begin{minipage}[b]{\linewidth}\raggedright
\includegraphics[width=2.87008in,height=2.40157in]{media/image23.png}

b) conventional (LCOH=26.53 USD/MWh)
\end{minipage} \\
\midrule\noalign{}
\endhead
\bottomrule\noalign{}
\endlastfoot
\end{longtable}

The heliostat field corresponds to around 36.5\% of the total cost and
for more than half of the equipment costs (this is, excluding land and
EPC costs), which is higher share than the conventional case (29.5\%).
Engineering and contingency (EPC) costs represent 23\%, and the storage
cost is around 19\%. The receiver cost (6.9\% for BDR, 20.1\% for
conventional) is the main difference between both cases, which is
actually one of the main uncertainties in the BDR cost estimation,
because the estimation for a free-falling design was used. A detailed
design and cost estimation for tilted-particle Receiver (BD-TPR) is then
a question that this research has open. The land is the 11.8\%
remaining, which means that the costs associated with the solar field
are roughly half of the total installing cost. The tower and BDR mirrors
have a minimal impact on the total cost (\textasciitilde2.8\% combined),
reflecting the benefits of the BDR concept, simplifying the tower
structure and costs. Finally, the novel design shows a 13\% reduction in
resulting LCOH (23.01 USD/MWh) compared with the conventional base case
(26.53 USD/MWh) which suggest that the proposed design offer
improvements in deployment costs.

These results have to be taken carefully and further refinement in these
calculations are required. Some factors can modify positively or
negatively these results. A factor not considered here is the financial
costs. No loans, debts, incentives, and taxes are included in either
case. In addition, the risk (represented by the discount rate) is kept
the same in both cases (5\%). Thus, the positive associated aspects of a
modular design are not reflected in the models. For example, the
expected reduction in construction times is not included. On other hand,
the proposed design is only \textasciitilde20 MW\textsubscript{th},
which is small for competitive power generation, due to the economy of
scale in power blocks. Several towers would be required to drive one
single power block, and the integration costs associated are also not
included. However, a multi-tower configuration would bring other
benefits, as it is the partial operation during the construction
process, like it has been reported in the commercial Chinese plant
reported in the Literature Review (Section \hl{2.3.4}) that started its
operation with only three of the 15 tower built {[}21{]}.

\subsubsection{Sensitivity analysis}\label{sensitivity-analysis}

An additional sensitivity analysis is performed over the main economic
parameters to analyse how they might change under different assumptions
or cost trends. The parameters are split in two groups, those
specifically related to the BD-TPR design (HB cost, TOD cost, receiver
cost, storage cost, tower cost, solar multiple, and storage capacity),
and those that are general for any CSP (heliostat costs, land costs,
O\&M costs, discount rate, lifetime, and EPC costs). Each of these
parameters is changed between an expected range and the impact over the
base LCOH is calculated. The main results are shown in \hl{}Figure 6.15.

It is observed that among those parameters specific of BD-TPR (Figure
6.15 a), the solar multiple is the factor that affects the most the
LCOH, followed by the storage capacity. A reduction of 50\% in solar
multiple increases the LCOH by 20\%, while duplicating the storage
capacity has also a 20\% impact in the final LCOH. The storage and
receiver costs have a similar impact over the LCOH. Doubling the cost
increases the LCOH around 10\%. The tower and BDR mirror (HB and TOD)
have almost no influence on the final cost, because they represent a
small fraction in the cost distribution, as it was discussed in the
previous section.

Among the general costs (Figure 6.15 b), the capacity factor (solar
field) (\(CF_{sf}\)) has the highest impact over the LCOH. The
relationship is inversely proportional: doubling the capacity factor
reduces the LCOH by half. The main factor that affects the capacity
factor is the solar resource, which explains why CSP feasibility depends
so much on the specific plant location. The engineering and contingency
costs is the second most influential parameter, which depends directly
on the maturity of the technology, as it was discussed in \hl{Section
(2.1.2)}. Solar tower's EPC costs are higher than PTC's as the
deployment knowledge is still progressing. It is expected that a
technology like BD-TPR would have a similar trend, with high initial EPC
costs, that should reduce once more plants are installed. The heliostat
cost and the discount rate are the next most influential factors. A
reduction in 50\% of heliostat field costs would mean a 15\% reduction
in LCOH. Plant lifetime and operation \& maintenance costs have a
significant influence too. It is clear from the comparison between both
plots in Figure 6.15 that the BDR-specific costs are less influential
than the general CSP costs, therefore, general trends observed in solar
tower technologies, should apply to BD-TPR plants too.

\begin{longtable}[]{@{}
  >{\centering\arraybackslash}p{(\linewidth - 0\tabcolsep) * \real{0.9989}}@{}}
\caption{Sensitivity Analysis results. a) BD-TPR specific
parameters, b) General parameters.}\tabularnewline
\toprule\noalign{}
\begin{minipage}[b]{\linewidth}\centering
\includegraphics[width=5.17539in,height=2.11303in]{media/image24.png}

a) BD-TPR parameters
\end{minipage} \\
\midrule\noalign{}
\endfirsthead
\toprule\noalign{}
\begin{minipage}[b]{\linewidth}\centering
\includegraphics[width=5.17539in,height=2.11303in]{media/image24.png}

a) BD-TPR parameters
\end{minipage} \\
\midrule\noalign{}
\endhead
\bottomrule\noalign{}
\endlastfoot
\includegraphics[width=5.34958in,height=2.18415in]{media/image25.png}

b) General CSP parameters \\
\end{longtable}

A probabilistic analysis was performed with these 14 parameters using a
Monte Carlo simulation with a total of 10\textsuperscript{6} runs. In
each run, each variable was assigned a random value between 0x and 2x
the default value, which corresponds with the range presented in Figure
6.15. A uniform distribution was chosen for each variable (all values in
the range are equally probable). It is important to note that, due to
computational time limitations, these runs were performed only on the
costing module (and not in the whole optimisation loop) using the
selected case shown in \hl{Section 6.4.2} as baseline. A LCOH
probabilistic distribution is obtained from this dataset and is
presented in \hl{}Figure 6.16.

The blue line corresponds to the probabilistic of non-exceedance, which
shows the probability to not exceed a given value of LCOH (X-axis). The
base case value is shown in grey, while P50 and P90 are shown in red.
P50 means that there is 50\% of probability that the LCOH would be
equals or lower than 22.7 USD/MWh, while the P90 line indicates that
there is a 90\% of probability that the LCOH would be equal or lower
than 49.9 USD/MWh. These results are useful inputs for decision makers
in order to improve the risk analysis of the technology.

\begin{figure}
\centering
\includegraphics[width=5.69756in,height=2.91991in]{media/image26.png}
\caption{Probability of non-exceedance for LCOH.
Probability curve in blue, baseline in grey, P50 and P90 in red.}
\end{figure}

\subsubsection{Annual performance}\label{annual-performance}

The expected performance through the year is explored in this section.
The overall efficiency considers the contribution of both the solar
field and receiver. The solar field optical efficiency depends mostly on
the sun position, which is usually represented by two main angles:
altitude (\(\alpha\), also known as elevation) and azimuth (\(z\)). A
set of datasets for different elevation and azimuth is generated and the
selected design is simulated with these sets.

\hl{(Write something about optical and thermal efficiencies through the
year)}

Using these two results for optical efficiency and receiver efficiency,
it is possible to analyse the performance of the system through the year
using the TMY file for the selected location (Alice Springs, Australia).
Figure 6. shows the annual performance grouped by month. The whole bar
indicates the total incident energy in the system, blue are the losses
in heliostat field, orange are the losses in the BDR, and green the
losses in the receiver. The stored useful energy is shown in red.
Subplot a) shows the absolute values for the 19MW\textsubscript{th}
plant, while the subplot b) shows the relative values, representing then
the component efficiencies. It can be seen that the overall efficiency
is relatively constant through the year with an average of 47\%. The
total annual generation is 44243 MWh, which corresponds to a 26.6\% of
capacity factor. The monthly average generation is 3686.9 MWh/month with
a variation in the range of \(( - 17.5\%,9.97\%)\).

\begin{figure}
\centering
\includegraphics[width=6.40952in,height=3.05208in]{media/image27.png}
\caption{Annual performance of the selected BD-TPR plant in
Alice Spring, Australia. Useful stored energy (red), heliostat losses
(blue), BDR losses (orange), and receiver losses (green). a) Absolute
energy, b) relative fraction (efficiencies).}
\label{fig:annual-performance}
\end{figure}

\hl{}Table 6.6 shows the annual performance values and are compared with
the conventional plant and the BD-TPR plant values in the design point.
In the case of the design point values, the annual production and
capacity factor are calculated assuming the overall efficiency is
constant through the year. It is observed a reduction of around
\textasciitilde10\% in annual production (and therefore capacity factor)
due to variations in different efficiencies. This justifies the factor
of 0.9 applied in the capacity factor, explained in Subsection 6.3.7,
therefore, the final LCOH is not affected. It has to be noted that most
of the change in efficiencies is due to heliostat efficiency (68.5\%
annual versus 74.2\% design point). The overall efficiency of a
conventional plant is higher than the BD-TPR by more than 10\%. However,
most of this difference is due to the receiver efficiency, that is
around 90\% for the conventional plant. If only the optic efficiencies
are considered, the difference is around 5\% (62.22\% for conventional
plant versus 68.5\% x 82.7\% = 56.6\%).

\begin{longtable}[]{@{}
  >{\raggedright\arraybackslash}p{(\linewidth - 6\tabcolsep) * \real{0.2499}}
  >{\centering\arraybackslash}p{(\linewidth - 6\tabcolsep) * \real{0.2500}}
  >{\centering\arraybackslash}p{(\linewidth - 6\tabcolsep) * \real{0.2500}}
  >{\centering\arraybackslash}p{(\linewidth - 6\tabcolsep) * \real{0.2500}}@{}}
\caption{Table 6.6. Results from annual performance compared with design
point average and conventional plant.}\tabularnewline
\toprule\noalign{}
\begin{minipage}[b]{\linewidth}\raggedright
\end{minipage} & \begin{minipage}[b]{\linewidth}\centering
BD-TPR annual
\end{minipage} & \begin{minipage}[b]{\linewidth}\centering
BD-TPR design point
\end{minipage} & \begin{minipage}[b]{\linewidth}\centering
Conventional plant
\end{minipage} \\
\midrule\noalign{}
\endfirsthead
\toprule\noalign{}
\begin{minipage}[b]{\linewidth}\raggedright
\end{minipage} & \begin{minipage}[b]{\linewidth}\centering
BD-TPR annual
\end{minipage} & \begin{minipage}[b]{\linewidth}\centering
BD-TPR design point
\end{minipage} & \begin{minipage}[b]{\linewidth}\centering
Conventional plant
\end{minipage} \\
\midrule\noalign{}
\endhead
\bottomrule\noalign{}
\endlastfoot
Nominal Capacity & 19 MW\textsubscript{th} & 19 MW\textsubscript{th} &
670 MW\textsubscript{th} \\
Annual Production & 44243 MWh & 49484 MWh & 12668 GWh \\
Capacity Factor & 26.5\% & 29.7\% & 21.6\% \\
Heliostats Efficiency & 68.5\% & 74.2\% & 62.2\% \\
BDR efficiency & 82.7\% & 85.4\% & - \\
Receiver Efficiency & 83.1\% & 83.1\% & 90.0\% \\
Overall Efficiency & 47.07\% & 52.65\% & 58.5\% \\
\end{longtable}

\subsection{Conclusion}\label{conclusion}

An optimisation tool is presented to find optimal parameters for a
BD-TPR plant. The optimisation tool put together the BDR optical
analysis developed in Chapter 3, the TPR thermal analysis developed in
Chapter 5, and the detailed costing estimations specifically for BDR
plants described here. All these separate modules are combined with an
in-house optimisation tool, which allows working with different
objective functions, including the minimisation of LCOH, which was used
here. This novel innovative tool together with the BD-TPR costing
calculations cover one of the research gaps found in the literature,
regarding the lack of knowledge about BDR economics and an
economic-based design for this technology.

The results show that for a given tower height, it is possible to find
optimal values of vertex ratio, receiver power, and average radiation
flux to minimise the LCOH. It was found that for a tower height range
between 35-50m, it is possible to obtain LCOH of \textasciitilde24
USD/MW\textsubscript{th} for receiver powers between 10 - 20
MW\textsubscript{th}. Specifically, for a 50 m tower, it was found that
a range of possible configurations generates a similar LOCH. In
particular, a 19 MW\textsubscript{th} with 1.25
MW\textsubscript{th}/m\textsuperscript{2} was selected. These results
are compared with a base case from the literature (LCOH = 26
USD/MW\textsubscript{th}) and a 13\% reduction in the LCOH was found. A
correlation for the expected LCOH for a given tower height was
presented, as well as the expected receiver power.

The detailed analysis on the selected design shows that it meets most of
the additional constraints of these plants, such as, limited size of
hyperboloid and TOD mirrors, adequate cooling requirement for BDR
mirrors, maximum radiation flux incident on the particles, reasonable
temperature distribution of the particles in the receiver outlet and
expected solar field and receiver efficiencies. The cost distribution
confirms the advantages of a beam-down configuration, reducing the
tower- and receiver-associated costs significantly, from
\textasciitilde27\% in the conventional case to \textasciitilde10\% of
the total capital cost. On other hand, this increases the share of the
heliostat field, which by itself is more than half of the total
component costs. A sensitivity analysis over the main economic
parameters shows that the capacity factor, solar multiple, and heliostat
costs are the most sensible parameter, together with the plant lifetime
and the EPC costs. A probability distribution using Monte Carlo
simulation was also presented. A P90 value of 49.9 USD/MWh and a P50
value of 23.0 USD/MWh were obtained.

The annual performance was also explored and compared with a
conventional case. A chart for the optical efficiency as function of the
sun position was developed with values between 50\% and 64\% (from a
design point efficiency of 63\%). The sun's position has higher
influence over the heliostats than the BDR mirror, mostly due to cosine
efficiency. The receiver efficiency is almost not affected by the sun's
position because the radiation flux remains relatively the same
regardless the elevation and azimuth. These different factors affect the
annual performance, with an 8\% reduction compared with the design point
case. The overall efficiency of a BD-TPR is lower than the conventional
one (47\% vs. 58\%), then the reduction in LCOH comes from higher
capacity factor (27\% vs. 22\%) and lower specific costs.

These findings suggest that BDRs coupled with solid particles as heat
transfer medium could be a viable alternative for modular CST plants to
drive small power generation plants or industrial heat applications. The
The performance of such system in real conditions is the next step in the
analysis, to be covered in Chapter \hl{7}.

\bibliographystyle{ieeetr}
\bibliography{../dwh/solarshift}

\end{document}
